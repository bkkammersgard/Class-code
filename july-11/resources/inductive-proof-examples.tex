\documentclass{article}
\usepackage[utf8]{inputenc}
\usepackage{amsmath}
\usepackage{amssymb}
\title{Inductive proof examples}
\begin{document}
\section{Sum from 1 to n}
We want to prove that \(1 + 2 + ... + n \) can be calculated with \( \frac{n(n+1)}{2} \)
\subsection{Definitions}
We will define a function to let us talk about the sum of numbers from 1 to $n$. Let:
\begin{equation}
F(n) = 1 + 2 + ... + n
\end{equation}
We will define a predicate to let us talk about the relationship between $F(n)$ and the shortcut calculation. Let:
\begin{equation}
P(n): F(n) = \frac{n(n + 1)}{2}
\end{equation}
Note that $P(n)$ evaluates to a boolean. It can be true or false for any particular $n$. It is true for a particular value of $n$ if $F(n)$ does in fact equal \( \frac{n(n + 1)}{2} \) and it is false if these two things are not equal.

\subsection{Goal}
Our goal is to prove that $P(n)$ holds (is true) for all values of $n$ greater than 0. Prove:
\begin{equation}
\forall n \in N : P(n)
\end{equation}

\subsection{Proof by induction}
\subsubsection{Base case}
To show our base case $P(1)$ is true, we will state the base case, then show that the left side does in fact equal the right side. Prove:
\begin{equation}
P(1): F(1) = \frac{1(1 + 1)}{2}
\end{equation}
\[ F(1) = 1 \]
\[ \frac{1(1 + 1)}{2} = \frac{2}{2} = 1 \]

\subsubsection{Inductive step}
We will prove that \textbf{if} $P(k)$ holds (is true) for some $k \in N$, \textbf{then} $P(k + 1)$ is also true. Prove:
\begin{equation}
P(k) \implies P(k + 1), \forall k \in N
\end{equation}
We start with the \textit{inductive hypothesis}, we assume for the time that $P(k)$ holds. Assume:
\begin{equation}
P(k): F(k) = \frac{k(k + 1)}{2}
\end{equation}
Now, assuming that $P(k)$ is true, prove:
\begin{equation}
P(k + 1): F(k + 1) = \frac{(k + 1)((k + 1) + 1)}{2}
\end{equation}
By definition:
\[ F(k + 1) = 1 + 2 + ... + k + (k + 1) \]
which is by definition:
\[ F(k + 1) = F(k) + (k + 1) \]
which by our inductive hypothesis is:
\[ F(k + 1) = \frac{k(k + 1)}{2} + (k + 1) \]
simplifying is:
\[ F(k + 1) = (k + 1) (\frac{k}{2} + 1) \]
which is equivalent to:
\[ F(k + 1) = (k + 1) (\frac{k}{2} + \frac{2}{2}) \]
which simplifies to:
\[ F(k + 1) = \frac{(k + 1) (k + 2)}{2} \]
which is clearly:
\[ F(k + 1) = \frac{(k + 1) ((k + 1) + 1)}{2} \]
And so we have proved $P(k + 1)$ (7) by showing that the left side is equal to the right side (assuming that $P(k)$ is true).
\subsection{Conclusion}
We have proved that $P(n)$ holds for a base case of $P(1)$ and that for all $k \in N$, $P(k)$ being true implies that $P(k + 1)$ is also true. Therefore $P(n)$ holds for all $n > 0$ (all natural numbers).
\[P(1): F(1) = \frac{1(1 + 1)}{2}\]
\[P(k) \implies P(k + 1), \forall k \in N \]
\[ \therefore P(n), \forall n \in N \]


\section{Making postage with 3 and 5 cent stamps}
We want to prove that all postage amounts greater than 7 cents can be made with combinations of 3 and 5 cent stamps

\subsection{Definitions}
We will define a predicate to talk about whether a particular number can be represented as a summation of a non-negative multiple of 3 and a non-negative multiple of 5.
\begin{equation}
P(n): n = 3a + 5b \mid a,b \in \mathbb Z_{\ge 0}
\end{equation}
Note that $P(n)$ may be true or false for any given number $n$. For example, $P(2)$ is false, as 2 cents of postage cannot be made with 3 and 5 cent stamps. However, $P(11)$ is true, because 11 cents of postage can be made with a 5 cent stamp and two 3 cent stamps.

\subsection{Goal}
Our goal is to prove that $P(n)$ holds for all values of $n$ greater than 7.
\begin{equation}
\forall n \in \mathbb Z_{> 7} : P(n)
\end{equation}

\subsection{Proof by induction}
\subsubsection{Base case}
To show that the base case $P(8)$ is true, we will state the base case, then show that we can find suitable non-negative integers $a$ and $b$. Prove:
\begin{equation}
P(8): 8 = 3a + 5b
\end{equation}
\[ a,b = 1 \]

\subsubsection{Inductive step}
We will prove that \textbf{if} $P(k)$ holds (is true) for some $k \in \mathbb Z_{> 7}$, \textbf{then} $P(k + 1)$ is also true. Prove:
\begin{equation}
P(k) \implies P(k + 1), \forall k \in \mathbb Z_{> 7}
\end{equation}
We start with the \textit{inductive hypothesis}, we assume for the time that $P(k)$ holds. Assume:
\begin{equation}
P(k): k = 3a + 5b \mid a,b \in \mathbb Z_{\ge 0}
\end{equation}
Now, assuming that $P(k)$ is true, prove:
\begin{equation}
P(k + 1): k + 1 = 3c + 5d \mid c,d \in \mathbb Z_{\ge 0}
\end{equation}
If $b > 0$, then $d = b - 1$ and $c = a + 2$ results in:
\[ 3c + 5d = 3(a + 2) + 5(b - 1) = 3a + 5b + 1 = k + 1 \]
By our inductive hypothesis, we assumed that $a$ and $b$ were non-negative integers, and so $c$ and $d$ will be too.

If $b = 0$, then $k$ must be a multiple of 3. The smallest multiple of 3 in $\mathbb Z_{> 7}$ is 9, which is $3 * 3$. All other multiples of 3 in $\mathbb Z_{> 7}$ will have a greater or equal number of 3 in their all-3 representation ($b = 0$). Therefore if $b = 0$, then $a \ge 3$. So $d = b + 2$ and $c = a - 3$ results in:
\[ 3c + 5d = 3(a - 3) + 5(b + 2) = 3a + 5b + 1 = k + 1 \]
By our inductive hypothesis and our reasoning that $a$ must be greater than or equal to 3 in the case where $b = 0$, then $c$ and $d$ will be non-negative integers too.

And so we have proved $P(k + 1)$ (13) by assuming $P(k)$ (12) was true. If $P(k)$, then $P(k + 1$ (11).

\subsection{Conclusion}
We have proved that $P(n)$ holds for a base case of $P(8)$ and that for all $k \in \mathbb Z_{> 7}$, $P(k)$ being true implies that $P(k + 1)$ is also true. Therefore $P(n)$ holds for all $n > 7$.
\[P(8): 8 = 3(1) + 5(1) \]
\[P(k) \implies P(k + 1), \forall k \in  \mathbb Z_{> 7}\]
\[ \therefore P(n), \forall n \in \mathbb Z_{> 7} \]


\section{Proof with inequality}
We want to prove that for any integer $n$ greater than 6, $n!$ is greater than $3^n$.
\[n! > 3^n \mid n > 6 \]
\subsection{Definitions}
We will define a predicate to let us talk about an inequality relationship between $n!$ and $3^n$. Let:
\begin{equation}
P(n): n! > 3^n
\end{equation}
Note that $P(n)$ may be true or false for any particular choice of $n$. For example, $P(2)$ is false because $2!$ is not greater than $3^2$. But $P(7)$ is true because $7!$ (5040) \textbf{is} greater than $3^7$ (2187).

\subsection{Goal}
Our goal is to prove that $P(n)$ holds for all values of $n$ greater than 6.
\begin{equation}
\forall n \in \mathbb Z_{> 6} : P(n)
\end{equation}

\subsection{Proof by induction}
\subsubsection{Base case}
To show that the base case $P(7)$ is true, we will state the base case, then show that the inequality is true. Prove:
\begin{equation}
P(7): 7! > 3^7
\end{equation}
\[ 7! = 5040 > 2187 = 3^7 \]

\subsubsection{Inductive step}
We will prove that \textbf{if} $P(k)$ holds (is true) for some $k \in \mathbb Z_{> 6}$, \textbf{then} $P(k + 1)$ is also true. Prove:
\begin{equation}
P(k) \implies P(k + 1), \forall k \in \mathbb Z_{> 6}
\end{equation}
We start with the \textit{inductive hypothesis}, we assume for the time that $P(k)$ holds. Assume:
\begin{equation}
P(k): k! > 3^k \mid k \in \mathbb Z_{> 6}
\end{equation}
Now, assuming that $P(k)$ is true, prove:
\begin{equation}
P(k + 1): (k + 1)! > 3^{k + 1}
\end{equation}
Starting with the left side:
\[ (k + 1)! = (k + 1) k! \]
Using our inductive hypothesis:
\[ (k + 1)! > (k + 1) 3^k \]
\[ (k + 1)! > \frac{1}{1} (k + 1) 3^k \]
\[ (k + 1)! > \frac{3}{3} (k + 1) 3^k \]
\[ (k + 1)! > \frac{k + 1}{3} 3 * 3^k \]
\[ (k + 1)! > \frac{k + 1}{3} 3^1 * 3^k \]
\[ (k + 1)! > \frac{k + 1}{3} 3^{k + 1} \]
And if $\frac{k + 1}{3}$ is greater than 1:
\[ (k + 1)! > \frac{k + 1}{3} 3^{k + 1} > 3^{k + 1} \]
therefore:
\[ (k + 1)! > 3^{k + 1} \]
Since $k \in \mathbb Z_{> 6}$, we know that $\frac{k + 1}{3}$ will always be greater than $\frac{6}{3}$ which is greater than 1.
And so we have proved $P(k + 1)$ (19) by assuming $P(k)$ (18) was true. If $P(k)$, then $P(k + 1$ (17).

\subsection{Conclusion}
We have proved that $P(n)$ holds for a base case of $P(7)$ and that for all $k \in \mathbb Z_{> 6}$, $P(k)$ being true implies that $P(k + 1)$ is also true. Therefore $P(n)$ holds for all $n > 6$.
\[P(7): 7! > 3^7 \]
\[P(k) \implies P(k + 1), \forall k \in  \mathbb Z_{> 6}\]
\[ \therefore P(n), \forall n \in \mathbb Z_{> 6} \]



\section{Another summation}
We want to prove that $ 1 + 4 + 7 + ... + (3n - 2) $ can be calculated with $ \frac{n(3n - 1)}{2} $



\end{document}
